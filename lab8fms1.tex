\documentclass {report}
\usepackage[T1,T2A]{fontenc}
\usepackage[utf8]{inputenc}
\usepackage[english,ukrainian]{babel}
\usepackage{listings}
\usepackage{verbatim}
\usepackage{graphicx}
\usepackage{oz,ztc}
\zedcompatible
\usepackage{graphicx}
\usepackage{makeidx}

\title{
\textbf{Міністерство освіти і науки України \\ 
Національний Авіаційний Університет \\
Факультет комп’ютерних наук та технологій\\}
\vspace{0.5cm}
\large 
\raggedleft
\textbf{Кафедра прикладної математики \\}
\vspace{1cm}
\center
\huge
\textbf{Звіт\\
\Large Про виконання лабораторної роботи \\
З предмету «Мови формальних специфікацій» \\
На тему: Складові схеми Z \\}
\vspace{0.3cm}
\raggedleft

\Large Виконав: \\
Студент 351 групи \\
Штуль В.С. \\
Перевірив:\\
Піскунов О. Г.
\center
\vspace{1cm}
м. Київ\\
2023
}
\date{}

\begin{document}

\maketitle
\titlepage
\tableofcontents
\newpage
\chapter{Вступ}

\section {Мета завдання}
Розробка схеми алгоритму єгипетського множення Ахмеса-Степанова.
\section {Постановка завдання}
• Додати до преамбули документа пакет та налаштування для коректного відображення символів мови Z (див. 2.5.1);\\

• Взяти за основу схеми алгоритму єгипетського множення Ахмеса-Степанова розроблені в [54, Схеми у стилі утиліти ZTC];\\

• Замінити у схемах тип intZ:\\
intZ == seq1 Char\\
\texttt{
forall n: intZ @ dom ( n rres \{'-', '+'\} ) subseteq \{1\}\\
and \# (n rres digits) > 0\\
тип цілих чисел Z;.\\}

• У алгоритмах, що записуються, використовувати операторний стиль запису замість функціонального, тобто замість sum(a, b) використовувати a + b;\\

• До звіту включити обидві версії специфікації: тестову версію та версію для LaTex.\cite{author1}\\

\chapter{Теоретична частина}

В алгоритмі єгипетського множення використовується вже розроблене додавання, функція поділу на два (half) та функція перевірки парності (odd) (schema Functions).\\

• функція toN відображає символ цифри у відповідне число;\\

 • частково певна функція toD відображає число від 0 до 9 у відповідну цифру;\\

• null – позначення відсутнього символу, аналогічно мові SQL. Тепер можна описати алгоритм поділу навпіл, з відкиданням дробової частини(schema  Half).


У поданій схемі:\\

•функція half видаляє знак з числа, виконує розподіл отриманого натурального числа навпіл за допомогою функції half0 на 2 і у разі знамінус дописує його до результату.\\

•функція half0 виконує розподіл цифри зі старшого розряду на 2 з урахуванням залишку від розподілу старшої цифри, передаючи цифри молодшихрозрядів і залишок від розподілу рекурсивному виклику себе.\\

•\texttt{функція half1 виконує розподіл числа (від 0 до 9) з урахуванням залишку a-відділення старшого розряду на 2 ((ch+a*10)/2). Додатково повертає залишок від свого поділу для молодшого розряду (i mod 2).}\\

Алгоритм функції парне - непарне очевидний і цікавий тільки з точки зору використання мови Z (schema Odd).\\

Тепер можна записати рекурсивну версію алгоритму множення Ахмеса-Степанова з акумулюванням (Schema Mult\_acc3).\\

Алгорим з акумулюванням добре працює на непарних числах, тому для парних чисел частину роботи можна робити без нього (Schema Multiply ).\\
\chapter{Практична частина}
\section {Розробка схеми алгоритму єгипетського множення Ахмеса-Степанова}
\input{res.zed}
\section {Розробка схеми алгоритму єгипетського множення Ахмеса-Степанова у текстовому вгляді}
\verbatiminput{res.zed}
\cite{author2}
\printindex
\cite{author1}
\renewcommand\bibname{Перелік джерел посилання}
\bibliographystyle{plain}
\bibliography{library}

\end{document}